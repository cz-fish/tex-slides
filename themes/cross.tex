%%% styl dokumentu - definuje, jak bude vypadat kazdy slajd po formalni
%%% strance a take jak budou vypadat jednotlive prvky

%%% makra pro praci s PostScriptem a PDF
\input pspdf

%%% definice spolecne pro vsechny styly
\input themes/common

%%% deklarace fontu
\font\titlefont=csssdc10 scaled\magstep4
\font\middlefont=csss9 scaled\magstep1
\font\textfont=csss9 scaled\magstep2
\font\bbulletfont=csss9 scaled\magstep1
\font\authorfont=csss9 scaled\magstep2
\font\headerfont=csss9
\font\codefont=cstt10

\textfont

%%% barvy
\def\hilitecolor{0.05 0.18 0.65}
\def\hhilitecolor{0.18 0.62 0.02}
\def\textcolor{0 0 0}
\def\pagenocolor{0 0 0}
\def\inactcolor{0.6 0.6 0.6}
\def\codecolor{0.35 0.35 0.35}

%%% indexy boxu s obrazky a textem
\newbox\imgboxright \newbox\imgboxleft \newbox\textbox \newbox\bulletbox
\newdimen\textboxwidth \newdimen\tmpwidth \newbox\tmpbox
\newdimen\maxboxsize \newdimen\tmpheight \newcount\oldpageno

% obrazek (soubor, velikost)
\def\image#1#2{\epsfxsize=#2\epsfbox{#1}}

% zvyraznene slovo
\def\hilite#1{\special{color rgb \hilitecolor}#1\special{color rgb \textcolor}}

% zvyraznene slovo jinou barvou
\def\hhilite#1{\special{color rgb \hhilitecolor}#1\special{color rgb \textcolor}}

% odrazka
\def\bullet#1{\ifx\firstbullet1\gdef\firstbullet{0}\else\vskip9pt\fi
\tmpwidth=\textboxwidth\advance\tmpwidth by-0.25in
% sestavit vbox z textu odrazky (muze byt pres vic radku)
\setbox\bulletbox=\vbox{\hsize=\tmpwidth{#1}}
% vyrobit z odrazky horizontalni box, samotny znak odrazky je ve vboxu,
% ktery ma velikost stejnou jako vbox textu (\ht0), aby se odrazka sazela
% pred prvni radek textu
\hbox{\hskip -12pt \vbox to\ht\bulletbox{\hsize=4mm\image{themes/cross_bul.eps}{3.2mm}\vfil}\box\bulletbox}}
% odrazka druhe urovne
\def\bbullet#1{\ifx\firstbullet1\gdef\firstbullet{0}\else\vskip4pt\fi
\tmpwidth=\textboxwidth\advance\tmpwidth by-0.75in
% ( 8in - 20pt ) / (1.2)^3 = 4.3in  ... 20pt ~ 0.5in; minus '*', hskip -> 4.15in
\setbox\bulletbox=\vbox{\hsize=\tmpwidth\bbulletfont{#1}}
\hbox{\hskip 24pt \vbox to\ht\bulletbox{\hsize=3mm\image{themes/cross_bbul.eps}{3mm}\vfil}\box\bulletbox}}

% neaktivni odrazky
% odrazka
\def\inactbul#1{\ifx\firstbullet1\gdef\firstbullet{0}\else\vskip9pt\fi
\tmpwidth=\textboxwidth\advance\tmpwidth by-0.25in
% sestavit vbox z textu odrazky (muze byt pres vic radku)
\setbox\bulletbox=\vbox{\hsize=\tmpwidth{\special{color rgb \inactcolor}#1\special{color rgb \textcolor}}}
% vyrobit z odrazky horizontalni box, samotny znak odrazky je ve vboxu,
% ktery ma velikost stejnou jako vbox textu (\ht0), aby se odrazka sazela
% pred prvni radek textu
\hbox{\hskip -12pt \vbox to\ht\bulletbox{\hsize=4mm\image{themes/cross_bul.eps}{3.2mm}\hfil\vfil}\box\bulletbox}}
% odrazka druhe urovne
\def\inactbbul#1{\ifx\firstbullet1\gdef\firstbullet{0}\else\vskip4pt\fi
\tmpwidth=\textboxwidth\advance\tmpwidth by-0.75in
% ( 8in - 20pt ) / (1.2)^3 = 4.3in  ... 20pt ~ 0.5in; minus '*', hskip -> 4.15in
\setbox\bulletbox=\vbox{\hsize=\tmpwidth\bbulletfont{\special{color rgb \inactcolor}#1\special{color rgb \textcolor}}}
\hbox{\hskip 24pt \vbox to\ht\bulletbox{\hsize=3mm\image{themes/cross_bbul.eps}{3mm}\hfil\vfil}\box\bulletbox}}

% kus kodu
\def\code#1{\ifx\firstbullet1\gdef\firstbullet{0}\else\vskip3pt\fi\hbox
{\codefont\special{color rgb \codecolor}#1\special{color rgb \textcolor}}}

% titulni stranka
\def\titlepage{
%%: offset .. posunuti vzhledem k aktualni pozici, 72bodu na palec
%%: llx lly urx ury .. lower-left, upper-right ... souradnice borderu .eps (lze precist z hlavicky)
%%: rhi rwi .. pozadovany rozmer obrazku, 720bodu na palec
%% hscale, vscale .. zmena meritka na procenta
%% angle, clip .. otoceni, povoleni orezani podle bounding boxu
%% hsize vsize .. orezavaci sirka, vyska

% pozadi titulni stranky
% kvuli magstep3 vychazi palec na 42 bodu misto 72
\special{psfile=themes/cross_title.eps hoffset=-42 voffset=-272 llx=0 lly=0 urx=900 ury=675 clip rhi=3150 rwi=4200}

\bgroup\nopagenumbers\pageno=0

\middlefont
~\vskip 0.65in
\special{color rgb \hilitecolor}
\centerline{\titlefont\Title}
\special{color rgb \textcolor}
\vskip 0.37in
\hbox to\hsize{\hfil\the\day.\the\month. \the\year\hskip0.48in}
\vskip 0.30in{\headerfont
\centerline{\Document}
\centerline{\Organisation}}
\vskip 0.34in
\special{color rgb \hilitecolor}
\hbox{\hskip0.6in\authorfont\Author\hfil}
\special{color rgb \textcolor}
\hbox{\hskip0.6in$<$\EMail$>$\hfil}

% konec stranky
\vfil
\penalty-10000
\egroup

% nastaveni dalsich stranek
% cisluj od 1
\pageno=1
% odstran footline
}

% zachovat cislo slajdu - tudiz nasledujici slajd bude mit stejne cislo jako
% predchozi
\def\keepslidenr{\pageno=\oldpageno}

% novy slajd (nadpis)
\def\slide#1{
% pozadi slajdu
\special{psfile=themes/cross.eps hoffset=-42 voffset=-272 llx=0 lly=0 urx=900 ury=675 clip rhi=3150 rwi=4200}
\oldpageno=\pageno
\bgroup
% nadpis slajdu
\vbox to0pt{\vskip-21pt\hskip-22pt\titlefont\hilite{#1}\vfil}
% navrat na zacatek strany, kde zacne vlastni obsah
\textfont\parindent=0pt\vskip9pt
% implicitni sirka textoveho boxu, pokud na slajdu nebudou vertikalni
% obrazkove boxy
\textboxwidth=4.6in
\gdef\firstbullet{1}
% text slajdu se bude ukladat do docasneho vboxu, ze ktereho se potom
% nasype do vysledneho vboxu pri sestavovani slajdu
% na konci slajdu se z nich sestavi vbox patricne sirky (pote co
% se vezmou do uvahy vertikalni boxy s obrazky)
\setbox\tmpbox=\vbox\bgroup}

% obrazky vlozene vedle sebe
\def\horizimages#1{\ifx\firstbullet1\gdef\firstbullet{0}\else\vskip9pt\fi\hbox to\textboxwidth{#1}}

% obrazku vlozene nad sebe na levou stranu
\def\vertimagesleft#1#2{
% odecist sirku obrazkoveho boxu od sirky textoveho boxu
\global\advance\textboxwidth by-#2
\global\advance\textboxwidth by-0.1in
% vytvorit obrazkovy box
\global\setbox\imgboxleft=\vbox{\hsize=#2{#1}}}

% obrazky vlozene nad sebe na pravou stranu
\def\vertimagesright#1#2{
% odecist sirku obrazkoveho boxu od sirky textoveho boxu
\global\advance\textboxwidth by-#2
\global\advance\textboxwidth by-0.03in
% vytvorit obrazkovy box
\global\setbox\imgboxright=\vbox{\hsize=#2{#1}}}

% vertikalni obrazkove boxy musi byt definovany jako prvni na celem
% slajdu (aby bylo mozno podle nich upravit sirku textoveho boxu)
% a po jejich skonceni musi byt vlozeno makro \vertimagesend
\def\vertimagesend{
% existujici textovy box bude ukoncen a bude zalozen novy uz se spravnou sirkou
\egroup
\setbox\tmpbox=\vbox\bgroup\hsize=\textboxwidth
}

% slozit textovy box
\def\producetextbox{\setbox\textbox=\vbox to\maxboxsize{\hsize=\textboxwidth\unvbox\tmpbox
\vfil}}

% konec slajdu
\def\endslide{
% ukonceni docasneho vboxu \tmpbox
\egroup
\ifdim\ht\tmpbox=0.0pt\setbox\tmpbox=\vbox{\hsize=\textboxwidth\vskip8pt
~}\fi
% zjistit vysku nejvyssiho z \imgboxleft, \imgboxright a \tmpbox
\maxboxsize=\ht\tmpbox
\ifvoid\imgboxleft\relax\else\ifdim\ht\imgboxleft>\maxboxsize\maxboxsize=\ht\imgboxleft\fi\fi
\ifvoid\imgboxright\relax\else\ifdim\ht\imgboxright>\maxboxsize\maxboxsize=\ht\imgboxright\fi\fi
\advance\maxboxsize by0.04in
% vytvorit vbox s textem (prelit \tmpbox do \textbox)
\producetextbox
% vysazet vboxy vedle sebe
\hbox{\hskip0.15in
\ifvoid\imgboxleft\relax\else\hskip-0.1in\vbox to\maxboxsize{\hsize=\wd\imgboxleft\unvbox\imgboxleft\vfil}\hskip0.2in\fi
\box\textbox
\ifvoid\imgboxright\relax\else\hskip0.03in\vbox to\maxboxsize{\hsize=\wd\imgboxright\unvbox\imgboxright\vfil}\fi}
% konec stranky
% paticka s nazvem prezentace a cislem strany
\tmpheight=2.94in\advance\tmpheight by-\maxboxsize
\vskip\tmpheight\hbox to4.9in{\hfil\special{color rgb \pagenocolor}\headerfont\Title\hfil\folio}
\vfill
\penalty-10000
\egroup
}


