%%% styl dokumentu - definuje, jak bude vypadat kazdy slajd po formalni
%%% strance a take jak budou vypadat jednotlive prvky

%%% makra pro praci s PostScriptem a PDF
\input pspdf

%%% definice spolecne pro vsechny styly
\input themes/common

%%% deklarace fontu
\font\titlefont=csss9 scaled\magstep4
\font\middlefont=csss9 scaled\magstep1
\font\textfont=csss9 scaled\magstep2
\font\bbulletfont=csss9 scaled\magstep1
\font\authorfont=csss9 scaled\magstep3
\font\headerfont=csss9
\font\codefont=cstt10

\textfont

%%% barvy
\def\hilitecolor{1 1 0}
\def\hhilitecolor{0 1 1}
\def\textcolor{0.7 0.7 0.7}
%\def\pagenocolor{0.75 0.345 0}
\def\pagenocolor{0.031 0.015 0.368}
\def\inactcolor{0.4 0.4 0.4}

%%% indexy boxu s obrazky a textem
\newbox\imgboxright \newbox\imgboxleft \newbox\textbox \newbox\bulletbox
\newdimen\textboxwidth \newdimen\tmpwidth \newbox\tmpbox
\newdimen\maxboxsize \newcount\oldpageno

% zvyraznene slovo
\def\hilite#1{\special{color rgb \hilitecolor}#1\special{color rgb \textcolor}}

% zvyraznene slovo jinou barvou
\def\hhilite#1{\special{color rgb \hhilitecolor}#1\special{color rgb \textcolor}}

% odrazka
\def\bullet#1{\ifx\firstbullet1\gdef\firstbullet{0}\else\vskip7pt\fi
\tmpwidth=\textboxwidth\advance\tmpwidth by-0.25in
% sestavit vbox z textu odrazky (muze byt pres vic radku)
\setbox\bulletbox=\vbox{\hsize=\tmpwidth{#1}}
% vyrobit z odrazky horizontalni box, samotny znak odrazky je ve vboxu,
% ktery ma velikost stejnou jako vbox textu (\ht0), aby se odrazka sazela
% pred prvni radek textu
\hbox{\hskip -12pt \vbox to\ht\bulletbox{\hsize=4mm\char126\vfil}\box\bulletbox}}
% odrazka druhe urovne
\def\bbullet#1{\ifx\firstbullet1\gdef\firstbullet{0}\else\vskip5pt\fi
\tmpwidth=\textboxwidth\advance\tmpwidth by-0.75in
% ( 8in - 20pt ) / (1.2)^3 = 4.3in  ... 20pt ~ 0.5in; minus '*', hskip -> 4.15in
\setbox\bulletbox=\vbox{\hsize=\tmpwidth\bbulletfont{#1}}
\hbox{\hskip 24pt \vbox to\ht\bulletbox{\hsize=4mm\bbulletfont*\vfil}\box\bulletbox}}

% neaktivni odrazky
% odrazka
\def\inactbul#1{\ifx\firstbullet1\gdef\firstbullet{0}\else\vskip7pt\fi
\tmpwidth=\textboxwidth\advance\tmpwidth by-0.25in
% sestavit vbox z textu odrazky (muze byt pres vic radku)
\setbox\bulletbox=\vbox{\hsize=\tmpwidth{\special{color rgb \inactcolor}#1\special{color rgb \textcolor}}}
% vyrobit z odrazky horizontalni box, samotny znak odrazky je ve vboxu,
% ktery ma velikost stejnou jako vbox textu (\ht0), aby se odrazka sazela
% pred prvni radek textu
\hbox{\hskip -12pt \vbox to\ht\bulletbox{\hsize=4mm\special{color rgb \inactcolor}\char126\special{color rgb \textcolor}\vfil}\box\bulletbox}}
% odrazka druhe urovne
\def\inactbbul#1{\ifx\firstbullet1\gdef\firstbullet{0}\else\vskip5pt\fi
\tmpwidth=\textboxwidth\advance\tmpwidth by-0.75in
% ( 8in - 20pt ) / (1.2)^3 = 4.3in  ... 20pt ~ 0.5in; minus '*', hskip -> 4.15in
\setbox\bulletbox=\vbox{\hsize=\tmpwidth\bbulletfont{\special{color rgb \inactcolor}#1\special{color rgb \textcolor}}}
\hbox{\hskip 24pt \vbox to\ht\bulletbox{\hsize=4mm\bbulletfont\special{color rgb \inactcolor}*\special{color rgb \textcolor}\vfil}\box\bulletbox}}

% kus kodu
\def\code#1{\vskip5pt\hbox{\codefont{}#1}}

% obrazek (soubor, velikost)
\def\image#1#2{\epsfxsize=#2\epsfbox{#1}}

% titulni stranka
\def\titlepage{
%%: offset .. posunuti vzhledem k aktualni pozici, 72bodu na palec
%%: llx lly urx ury .. lower-left, upper-right ... souradnice borderu .eps (lze precist z hlavicky)
%%: rhi rwi .. pozadovany rozmer obrazku, 720bodu na palec
%% hscale, vscale .. zmena meritka na procenta
%% angle, clip .. otoceni, povoleni orezani podle bounding boxu
%% hsize vsize .. orezavaci sirka, vyska

% pozadi titulni stranky
% kvuli magstep3 vychazi palec na 42 bodu misto 72
\special{psfile=themes/vim_title.eps hoffset=-42 voffset=-272 llx=0 lly=0 urx=721 ury=541 clip rhi=3150 rwi=4200}

\bgroup\nopagenumbers\pageno=0

\middlefont
\special{color rgb \hilitecolor}
\centerline{\titlefont\Title}
\special{color rgb \textcolor}
\vskip 0.15in
\centerline{\the\day.\the\month. \the\year}
\vskip 0.25in
\special{color rgb \hilitecolor}
\centerline{\authorfont\Author}
\special{color rgb \textcolor}
\centerline{$<$\EMail$>$}
\vskip 0.25in
\centerline{\Document}
\centerline{\Organisation}

% konec stranky
\vfil
\penalty-10000
\egroup

% nastaveni dalsich stranek
% cisluj od 1
\pageno=1
% odstran footline
}

% zachovat cislo slajdu - tudiz nasledujici slajd bude mit stejne cislo jako
% predchozi
\def\keepslidenr{\pageno=\oldpageno}

% novy slajd (nadpis)
\def\slide#1{
%\footline={\hfil\special{color rgb 1 0 0}blablabla}
%\headline={\special{color rgb 1 0 0}xxblaxx\hfil}
% pozadi slajdu
\special{psfile=themes/vim.eps hoffset=-42 voffset=-272 llx=0 lly=0 urx=721 ury=541 clip rhi=3150 rwi=4200}
\oldpageno=\pageno
\bgroup
\vbox to0pt{\vskip-51.5pt\hskip-57pt\headerfont\special{color rgb \pagenocolor}\folio.
\hskip3pt \Title\vfil}
\vbox to0pt{\vskip-45pt\centerline{\titlefont\hilite{#1}}\vfil}
\textfont\parindent=0pt\vskip 3pt
% implicitni sirka textoveho boxu, pokud na slajdu nebudou vertikalni
% obrazkove boxy
\textboxwidth=4.8in
\gdef\firstbullet{1}
% text slajdu se bude ukladat do docasneho vboxu, ze ktereho se potom
% nasype do vysledneho vboxu pri sestavovani slajdu
% na konci slajdu se z nich sestavi vbox patricne sirky (pote co
% se vezmou do uvahy vertikalni boxy s obrazky)
\setbox\tmpbox=\vbox\bgroup}

% obrazky vlozene vedle sebe
\def\horizimages#1{\vskip7pt\hbox to\textboxwidth{#1}}

% obrazku vlozene nad sebe na levou stranu
\def\vertimagesleft#1#2{
% odecist sirku obrazkoveho boxu od sirky textoveho boxu
\global\advance\textboxwidth by-#2
%\global\advance\textboxwidth by-0.2in
% vytvorit obrazkovy box
\global\setbox\imgboxleft=\vbox{\hsize=#2{#1}}}

% obrazky vlozene nad sebe na pravou stranu
\def\vertimagesright#1#2{
% odecist sirku obrazkoveho boxu od sirky textoveho boxu
\global\advance\textboxwidth by-#2
\global\advance\textboxwidth by-0.1in
% vytvorit obrazkovy box
\global\setbox\imgboxright=\vbox{\hsize=#2{#1}}}

% vertikalni obrazkove boxy musi byt definovany jako prvni na celem
% slajdu (aby bylo mozno podle nich upravit sirku textoveho boxu)
% a po jejich skonceni musi byt vlozeno makro \vertimagesend
\def\vertimagesend{
% existujici textovy box bude ukoncen a bude zalozen novy uz se spravnou sirkou
\egroup
%%%%\def\textstart{
\setbox\tmpbox=\vbox\bgroup\hsize=\textboxwidth
}

% slozit textovy box
\def\producetextbox{\setbox\textbox=\vbox to\maxboxsize{\hsize=\textboxwidth\unvbox\tmpbox
%\ifvoid\imgboxleft{\vskip 6pt levy box je prazdny \the\textboxwidth}\else
%{\vskip6pt levy box neni prazdny \the\textboxwidth~\expandafter\the\wd\imgboxleft}\fi
%\ifvoid\imgboxright{\vskip 6pt pravy box je prazdny \the\textboxwidth}\else
%{\vskip6pt pravy box neni prazdny \the\textboxwidth~\expandafter\the\wd\imgboxright}\fi
\vfil}
}

% konec slajdu
\def\endslide{
%%%%\ifvoid\imgboxleft\relax\else{\vskip6pt levy box neni prazdny \the\hsize \expandafter\the\wd\imgboxleft}\fi
%%%%\ifvoid\imgboxright\relax\else{\vskip6pt pravy box neni prazdny \the\hsize}\fi
% ukonceni docasneho vboxu \tmpbox
\egroup
% zjistit vysku nejvyssiho z \imgboxleft, \imgboxright a \tmpbox
\maxboxsize=\ht\tmpbox
\ifvoid\imgboxleft\relax\else\ifdim\ht\imgboxleft>\maxboxsize\maxboxsize=\ht\imgboxleft\fi\fi
\ifvoid\imgboxright\relax\else\ifdim\ht\imgboxright>\maxboxsize\maxboxsize=\ht\imgboxright\fi\fi
\advance\maxboxsize by0.04in
% vytvorit vbox s textem (prelit \tmpbox do \textbox)
\producetextbox
% vysazet vboxy vedle sebe
\hbox{\ifvoid\imgboxleft\relax\else\hskip-0.1in\vbox to\maxboxsize{\unvbox\imgboxleft\vfil}\hskip0.2in\fi
\box\textbox
\ifvoid\imgboxright\relax\else\hskip0.2in\vbox to\maxboxsize{\unvbox\imgboxright\vfil}\fi}
% konec stranky
\vfill
\penalty-10000
\egroup
}


